\documentclass[uplatex,a4j,12pt,twocolumn]{jsarticle}

\renewcommand{\baselinestretch}{0.99}
% 縦横のサイズ調整
\usepackage[margin=10mm]{geometry}
\usepackage[dvipdfmx]{graphicx,hyperref}
\usepackage{pxjahyper}
\usepackage{latexsym}
\usepackage{bmpsize}
\usepackage{url} % urlを参考文献で出力したいから使う
\usepackage{comment}
\usepackage{textcomp} % > などの記号を出力
\usepackage{here} % 画像を強制的に指定した箇所に出力
\usepackage{minted} % ソースコードのシンタックスハイライト
\usepackage{mdframed}
% 図の上下に微妙な隙間が出るので対策
\setlength\intextsep{3pt}
\setlength\textfloatsep{0pt}

\renewcommand{\listingscaption}{コード}  % キャプション名をコードに変更
\begin{document}

\title{\bf{\LARGE{Sample of \LaTeX  Document} \\ \Large{\LaTeX のサンプルコード}}}
\author{平木場 風太\\宮崎大学大学院}
\date{2019年6月12日(水)}
\maketitle


\section{はじめに}
\begin{figure}[t]
    \begin{center}
        \includegraphics[width=7cm]{image/syokuji_computer.png}
        \caption{パソコンの前でご飯を食べる人のイラスト}
        \label{fig:syokuji_computer}
    \end{center}
\end{figure}

パソコンの前でご飯を食べることはよくある。パソコンの前でご飯を食べる人のイラストを図\ref{fig:syokuji_computer}に示す。
このイラストは、規約の範囲内であれば、個人、法人、商用、非商用問わず無料で利用できることでおなじみの、{\bf かわいいフリー素材 いらすとや}より引用した。\cite{bib:console}引用です。コード\ref{listings:fastlane}を示す。

\begin{listing}[h]
    \begin{minted}[breaklines]{ruby}
  default_platform(:mac)
  platform :mac do
    desc "Build macOS App"
    lane :app_build do
    ime_path = "/Library/Input Methods"
      build_mac_app(
      configuration: "Debug",
      export_method: "development",
      scheme: "Raelize",
      output_directory: "#{ENV["HOME"]}#{ime_path}",
      clean: true
      )
      sh("pkill", "Raelize")
    end
  end
    \end{minted}
    \caption{fastlaneを使ったビルド自動化}\label{listings:fastlane}
  \end{listing}

\section{おわりに}
やっぱり{\bf いらすとや}のイラストはすばらしい。

%参考文献
\bibliography{sample} 
\bibliographystyle{junsrt} 

\end{document}
